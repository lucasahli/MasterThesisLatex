%%%%%%%%%%%%%%%%%%%%%%%%%%%%%%%%%%%%%%%%%%%%%%%%%%%%%%%%%%%%%%%%%%%%%%%%
% This is the conclusion chapter file.
%%%%%%%%%%%%%%%%%%%%%%%%%%%%%%%%%%%%%%%%%%%%%%%%%%%%%%%%%%%%%%%%%%%%%%%%
%
% Author:   René Widmer
%           Institute for Surgical Technology and Biomechanics ISTB
%           University of Bern
%           rene.widmer@istb.unibe.ch
%
% Date:     10/28/2009
%
%%%%%%%%%%%%%%%%%%%%%%%%%%%%%%%%%%%%%%%%%%%%%%%%%%%%%%%%%%%%%%%%%%%%%%%%

\begin{abstract}
\noindent\textbf{Introduction} Patients with liver cancer are only curatively treated by
surgery. Liver resection surgeries are the gold standard to treat these
patients. Image guided navigation approaches are not regularly used in this kind
of surgery because the involved registration of preoperative imaging data to the
surgical site leads to difficulties with deformations. Therefore the aim of this
thesis is to conceptualize, implement and evaluate a new approach to navigate in
tissue sparing resections which works without registration. This method should
intraoperatively reconstruct the liver and create a surgical plan to resect
tumors near the liver surface. 
\vskip1em
\noindent\textbf{Methods} The reconstruction of the liver is done in two steps. First,
the surface is reconstructed from a sample of points which were taken from
tracked ultrasound after analysing whether the ultrasound probe is on the liver
surface or not. Secondly, the tumor is reconstructed as a sphere. The location
and diameter of this sphere are determined from the tracked ultrasound images.
After freezing an ultrasound image which shows the largest diameter of the
tumor, it is semi-automatically segmented. The center of the segmentation is the
tumor center and the diameter is estimated from the segmentation contour. Using
the created 3D liver model a plan to resect the tumor is created by fitting a
resection shape which respects the safety margin around the tumor. Finally, to
test the resulting approach, two experiments are performed. One experiment
to evaluate the accuracy of the reconstructed surface was done on a liver
phantom. A usability test was conducted with three surgeons as an experiment to evaluate the
software applicability in the operating room. 
\vskip1em
\noindent\textbf{Results} The accuracy experiment of the reconstructed liver surface
shows a median error of 2.57 mm. Most of the accessible liver surface has been
reconstructed exactly but large errors are visible at the edge of
the reconstructed surface. The usability test shows that the surgeons find the
software useful and that they could imagine to use such a system in the
operating room.
\vskip1em
\noindent\textbf{Conclusions} 
A method to create an intraoperative model of the liver which can be used to
plan resections of tumors near the surface was developed. Experiments to
evaluate the method show results that are good in this prototype status and look promising for the
future development of this project. It has to be further tested if such an
approach is applicable in clinics. 

% \noindent\textit{The abstract should provide a concise (300-400 word) summary of the motivation, methodology, main results and conclusions. For example:}

% \vskip1em

% Osteoporosis is a disease in which the density and quality of bone are reduced. As the bones become more porous and fragile, the risk of fracture is greatly increased. The loss of bone occurs progressively, often there are no symptoms until the first fracture occurs. Nowadays as many women are dying from osteoporosis as from breast cancer. Moreover it has been estimated that yearly costs arising from osteoporotic fractures alone in Europe worth 30 billion Euros.

% Percutaneous vertebroplasty is the injection of bone cement into the vertebral body in order to relieve pain and stabilize fractured and/or osteoporotic vertebrae with immediate improvement of the symptoms. Treatment risks and complications include those related to needle placement, infection, bleeding and cement extravazation. The cement can leak into extraosseous tissues, including the epidural or paravertebral venous system eventually ending in pulmonary embolism and death.

% The aim of this project was to develop a computational model to simulate the flow of two immiscible fluids through porous trabecular bone in order to predict the three-dimensional spreading patterns developing from the cement injection and minimize the risk of cement extravazation while maximizing the mechanical effect. The computational model estimates region specific porosity and anisotropic permeability from Hounsfield unit values obtained from patient-specific clinical computer tomography data sets. The creeping flow through the porous matrix is governed by a modified version of Darcy's Law, an empirical relation of the pressure gradient to the flow velocity with consideration of the complex rheological properties of bone cement.

% To simulate the immiscible two phase fluid flow, i.e. the displacement of a biofluid by a biomaterial, a fluid interface tracking algorithm with mixed boundary representation has been developed. The nonlinear partial differential equation arising from the problem was numerically implemented into the open-source Finite Element framework \textit{libMesh}. The algorithm design allows the incorporation of the developed methods into a larger simulation of vertebral bone augmentation for pre-surgical planning.

% First simulation trials showed close agreement with the findings from relevant literature. The computational model demonstrated efficiency and numerical stability. The future model development may incorporate the morphology of the region specific trabecular bone structure improving the models' accuracy or the prediction of the orientation and alignment of fiber-reinforced bone cements in order to increase fracture-resistance. 

\end{abstract}

\endinput
%%% Local Variables:
%%% TeX-master: "MscThesis"
%%% End: