\chapter{State of the art}

\section{Intraoperative ultrasound}
Ultrasound imaging works by the \textit{pulse-echo} principle. A short
ultrasound-pulse is emitted from a transducer. Then the soundwaves get
transmitted and reflected differently by different tissues. The reflected
soundwaves travel back into the transducer and get converted into an electrical
signal. After post-processing these signals become ultrasound images. Basically
the ultrasound measures the mechanical properties of the tissue. The tissues
have different acoustic impedance, which is the product of tissue density and
ultrasound speed in travelling through the tissue. The resolution of the
ultrasound images depends on the frequency of the ultrasound waves. High
frequencies lead to high resolutions but low depth into the tissue because the
absorption of the sound energy increases with frequency too. Therefore the
useability to see deep structures is limited \cite{torzilli2014ultrasound}.

\section{Navigation for liver resections}
Navigation in liver surgeries is mostly done by registering the patient to a pre-operative 3D
computer tomography (CT) scan of the liver during the surgery. The achieved
navigation accuracy was $4.5 mm \pm3.6 mm$ \cite{peterhans2011navigation}.

\subsection{Registration methods}
\subsection{Tracking modalities}
\subsubsection{Optical tracking}
\subsubsection{Electromagnetic tracking}
\section{Surface reconstruction}
\section{Others}

\endinput
%%% Local Variables:
%%% TeX-master: "MscThesis"
%%% End: