\chapter{State of the art}

\section{Intraoperative ultrasound}
Ultrasound imaging works by the \textit{pulse-echo} principle. A short
ultrasound-pulse is emitted from a transducer. Then the soundwaves get
transmitted and reflected differently by different tissues. The reflected
soundwaves travel back into the transducer and get converted into an electrical
signal. After post-processing these signals become ultrasound images. Basically
the ultrasound measures the mechanical properties of the tissue. The tissues
have different acoustic impedance, which is the product of tissue density and
ultrasound speed in travelling through the tissue. The resolution of the
ultrasound images depends on the frequency of the ultrasound waves. High
frequencies lead to high resolutions but low depth into the tissue because the
absorption of the sound energy increases with frequency too. Therefore the
useability to see deep structures is limited \cite{torzilli2014ultrasound}.

\section{Navigation for liver resections}
Navigation in liver surgeries is mostly done by registering the patient to a pre-operative 3D
computer tomography (CT) scan of the liver during the surgery. All surgical
instruments have trackable markers attached to them. A tracking camera sees
these markers and can differentiate the different instruments from their attached
markers. The achieved
navigation accuracy was $4.5 mm \pm3.6 mm$ averaged over nine surgeries \cite{peterhans2011navigation}.
Current research tries to compensate for deformations of the liver after the CT
scan to the actual shape \cite{clements2017deformation}
\cite{clements2015validation}. 

\subsection{Registration methods}
Different registration methods exist. Discrete landmarks, surface scans and
volumetric sonography scans are just a few of the approaches that can be
used to achieve precise alignment of the preoperative image data with the
surgical site \cite{banz2016intraoperative}.

\subsection{Tracking modalities}
To track (define the position and orientation in real time) surgical instruments
and patient's anatomy during naviagated surgery a tracking system is needed.
Tracking can be done by different technologies. The most used tracking
modalities are optical or electromagnetic tracking. 

\subsubsection{Optical tracking}
Optical tracking is the most used tracking modality in naviagated liver
surgeries. Passive markers (spherical, retro-reflective that reflect infrared
light) or active markers (infrared-emitting markers that are activated by an
electrical signal) \cite{wiles2004accuracy} are attached to the objects that
need to be tracked. A tracking camera is then emitting infrared light by illuminators
on the position sensor (only for passive markers). The position sensor
determines the position and orientation of the tracked instruments based on the
information it receives from those markers.  \cite{noauthor_polaris_nodate}
\subsubsection{Electromagnetic tracking}
\section{Surface reconstruction}
\section{Others}

\endinput
%%% Local Variables:
%%% TeX-master: "MscThesis"
%%% End: