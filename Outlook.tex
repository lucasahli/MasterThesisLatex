%%%%%%%%%%%%%%%%%%%%%%%%%%%%%%%%%%%%%%%%%%%%%%%%%%%%%%%%%%%%%%%%%%%%%%%%
% This is the outlook chapter file.
%%%%%%%%%%%%%%%%%%%%%%%%%%%%%%%%%%%%%%%%%%%%%%%%%%%%%%%%%%%%%%%%%%%%%%%%
%
% Author:   René Widmer
%           Institute for Surgical Technology and Biomechanics ISTB
%           University of Bern
%           rene.widmer@istb.unibe.ch
%
% Date:     10/28/2009
%
%%%%%%%%%%%%%%%%%%%%%%%%%%%%%%%%%%%%%%%%%%%%%%%%%%%%%%%%%%%%%%%%%%%%%%%%
% from Hoppe \cite{hoppe1992surface} 
\chapter{Outlook}
So far the software was only tested on a liver phantom. Further investigations
should be done to prove the applicability of the method in real patients.
Therefore the software should be tested on an ex-vivo liver. Such a test could
involve the removal of an artificially placed tumor and an accuracy measurement
of the surface. However, also an evaluation with real patient data from a
navigated liver resection could help to further analyze the correct functioning
of the liver surface detector and the tumor segmentations on the ultrasound
images. Additionally the liver surface detector should be extended to be able to
distinguish different organs such that it does not recognize other organ's
surfaces as liver surfaces.

In order to be able to navigate laproscopic resections with this method as well,
the possibility to track electromagnetically should be integrated. This would
lead to the method being tested and used more frequently as most resections are performed laproscopiclly.

% Surface reconstruction on an ex-vivo liver --> test cutting out a tumor, measure accuracy
% Evaluation on real patient data from navigated liver resection --> surface
% detection, tumor segmentation
% EM tracking --> enables for more surgeries that can be used for testing

Future work to continue and extend the developed concept should go in the
direction of completing the 3D intraoperative reconstruction. The liver's vessel
structures should be added to the model. Probably as a first step only the
vessels near or in the volume affected by the planned resection.

In the current version of the software, tumors are approximated by spheres.
This should be improved by adding a method to accurately reconstruct tumors
automatically after semi automatic initialization. This would lead to more accurate
respectation of safety margins around them. As a first step the semi automatic
segmentation method should be optimized such that it works without manual
corrections after the first manual initialization.

At the moment only one resection shape can be selected and it cannot be adapted.
A method to adjust the resection's shape, surface entering location and safety
margin would help the surgeons to create better plannings and should be added to
surface. If adjusting one shape is not enough for the surgeons needs, then more
shapes should added. 

% Reconstruct accurate shape of the tumor to improve accurate safety margin respectation.
% Surface reconstruction on an ex-vivo liver
% Evaluation on real data from navigated liver resection
% Inclusion of vessel structures
% Adjust resection plan intraoperatively
% \textit{Provide a vision of possible future work to continue and extend your thesis research.}

% \cite{postelnicu2009combined} paper -->Combined volumetric and surface registration

\endinput
%%% Local Variables:
%%% TeX-master: "MscThesis"
%%% End: