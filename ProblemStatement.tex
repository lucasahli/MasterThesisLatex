\chapter{Problem Statement}
\textcolor{green}{start with the clinical problem that you want to solve}
\textcolor{green}{why do we need navigation then, and why do we need yet another method?}
Liver cancer is the fifth deadliest cancer in the world \cite{bray2018global}.
Primary hepatocellular carcinoma (HCC) and metastatic colorectal hepatic
metastasis (CHRM) are the most common hepatic tumours encountered \cite{north2014microwave}. Worldwide
there are 600'000 liver cancer deaths and 700'000 new cases each year.
The 5-year survival rate is very low, smaller than 20\% for these
patients. In Switzerland, around 700 people are diagnosed every year with
primary liver cancer. Still, more frequent than primary liver tumours are
secondary ones; those that spread as metastases from other organs such as stomach, colon,
lungs and breast. 

The only way to curatively treat patients with liver tumors is by surgery. Three types of surgery
are performed regularly on patients:
\begin{itemize}
  \item Anatomical resections
  \item Parenchymal-sparing (non-anatomical) resections
  \item Tumor ablations 
\end{itemize}
The goal of these treatments is the destruction or removal of the tumor. If the
tumor is not visible from the outside of the liver, then it is difficult for the
surgeons to tell where exactly it is. Therefore computer assisted navigation
could help to locate and visualize it intraoperatively.
In the case of liver resection surgeries, navigation systems are rarely employed
since they do not provide enough benefits to justify the additional time needed
to set them up \cite{beller2007image}. These two problems have several origins.
First, there is an accuracy problem for registration based approaches. The
registration error and the deformations of the liver during the surgery
\cite{clements2017deformation} or between the preoperative scan and the surgery
lead to errors. Secondly, the time needed to register the patient's anatomy to the
preoperative 3D-model. These methods need sometimes multiple attempts to complete
a sufficiently accurate registration. Moreover preoperative 3D-models are time
consuming and therefore expensive to generate.

In conclusion we aim to develop a new navigation concept based on intraoperative
imaging to make computer assisted navigation in liver resection surgeries more
accessible to liver surgeons. 
% This concept should not need a preoperative scan and consequently
% not need a registration.
% That way we would also avoid the preoperative 3D-model. 




% \textcolor{red}{ you're building a very specific application for non-anatomical resections. the bullet points should not be the requirements of your softwre but rather of your thesis}
% To make computer assisted navigation in liver resection surgeries more
% accessible to liver surgeons, a new concept has to be developed in order to do
% some first testings.
% Specifically, the method should fulfill the following:
% \begin{itemize}
%   \item The software should guide the surgeon through the surgical procedure.
%   \item A 3D-model of the liver should be created during the surgery.
%   \item The planning for the resection of the tumor should be done
%     intraoperatively on the created 3D-model.
%   \item The method should be ready for testings in the OR.
% \end{itemize}
\endinput
%%% Local Variables:
%%% TeX-master: "MscThesis"
%%% End: