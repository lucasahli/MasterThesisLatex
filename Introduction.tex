%%%%%%%%%%%%%%%%%%%%%%%%%%%%%%%%%%%%%%%%%%%%%%%%%%%%%%%%%%%%%%%%%%%%%%%%
% This is the introduction chapter file.
%%%%%%%%%%%%%%%%%%%%%%%%%%%%%%%%%%%%%%%%%%%%%%%%%%%%%%%%%%%%%%%%%%%%%%%%
%
% Author:   René Widmer
%           Institute for Surgical Technology and Biomechanics ISTB
%           University of Bern
%           rene.widmer@istb.unibe.ch
%
% Date:     10/28/2009
%
%%%%%%%%%%%%%%%%%%%%%%%%%%%%%%%%%%%%%%%%%%%%%%%%%%%%%%%%%%%%%%%%%%%%%%%%

\chapter{Introduction}
\section{Motivation} 
\section{The Liver} 
\subsection{Liver Anatomy}
The human liver overlies the gallbladder, is located in the right upper quadrant of the abdomen and has
different functions. It produces biochemicals necessary for digestion,
synthesizes proteins and detoxifies various metabolites. A human liver wheighs
normally around $1.5 kg$, is the heaviest internal organ and the largest gland
of the human body. Two large blood vessels are connected to the liver: the
portal vein and the hepatic artery. Both of them subdivide into small
capillaries called \textit{liver sinusoids} and then lead to the functional
units of the liver known as \textit{lobules}. To refer to the different parts of
the liver, it is subdivided into eight subsegments. Each segment has its own
vascular inflow and outflow.
\subsection{Liver Cancer}
\section{Liver Resections} 
\section{Objectives} 

\endinput
%%% Local Variables:
%%% TeX-master: "MscThesis"
%%% End: