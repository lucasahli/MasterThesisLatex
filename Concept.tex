\chapter{Concept}
In this chapter the desired concept will be presented. 
\section{System}
The hardware used with this system consists of:
\begin{itemize}
  \item a tracked ultrasound device
  \item a tracked pointer tool
  \item an optical tracking camera to track the instruments
  \item a computer to run the software
  \item a 3D-monitor which displays the 3D contents of the software 
  \item a touch-screen on a 2D-monitor to operate the software and show the
    ultrasound images
\end{itemize}
The software in this system consists of:
\begin{itemize}
  \item a sampling method to collect points on the liver-surface
  \item a reconstruction method to reconstruct the surface from the sampled points
  \item a segmentation method to segment the tumors on the ultrasound images
  \item a planning method to plan the resection of the liver
  \item a navigation mode used to navigate during the removal of the tumor
\end{itemize}
\section{Functionalities}
The three main functionalities of the developed concept will be presented in
this chapter. These functionalities were specifically developed for this project.
\subsection{Surface Reconstruction}
During surgery ultrasound images and their corresponding 6D poses (positions and
orientations) are collected and analyzed. First each ultrasound image has to be
checked for contact with the liver. If the ultrasound passes the check, that
means the ultrasound image looks like an ultrasound image that can only arise
when the ultrasound probe lies on the liver surface, then the position of this
image can be used.

In order to use the sampled position corresponding to an image, this
position has to be transformed into the correct coordinate system first. There
are four different coordinate systems. The first coordinate system is the image
coordinate system. The units in the image coordinate system are pixels and the
origin is in the top left corner of the image. The
second coordinate system is the ultrasound coordinate system. The origin of this
coordinate system is at the probe tip in the middle and the units in this and
the following coordinate systems are millimeters. The third coordinate system is the
ultrasound-tool-marker coordinate system. The origin is ... . The final
coordinate system is the tracking camera coordinate system. The origin of this
coordinate system is at the position sensor in the tracking camera and can not
be changed.

At the end of this transformation chain, a image pixel 2D position was
transformed into a tracking camera 3D positon and the units changed
from pixel to millimeter. This 3D location in the tracking camera coordinate system
will be added to the collection of points to later reconstruct the surface from.

After collecting the surface points, the reconstruction algorithm from Hoppe
\cite{hoppe1992surface} reconstructs the surface from these points.
\subsection{Tumor Segmentation}
To reconstruct and later plan the resection of a tumor, the shape
of the tumor has to be made visible first. Because most liver tumors are not visible from the outside of the liver, an
ultrasound device is mostly used during liver resections to look behind the
liver surface.
Most tumors have roundish shapes and a sphere is the easiest
geometrical shape that can be used to approximate a tumor's real shape. To
define a sphere two components are needed: the location and the radius of the
spere.
To find the location of the tumor, the surgeon locates the tumor with the
ultrasound. Then he freezes the ultrasound image that cuts through the middle of
the tumor. The 6D pose of that ultrasound image is stored and the image is
passed to the next step. The tumor on the image has to be segmented. This
segmentation is done semi automatically. That means the surgeon has to
roughly initialize the segmentation manually and then the graph cut algorithm
implemented by openCV will segment the the tumor. From the resulting
segmentation shape, the tumor center and radius are estimated. The center
corresponds to the mean of the segmented boarder pixels and the radius is the
mean between the largest and the shortest distance from the boarder pixels to
the estimated center pixel. By using the 6D pose corresponding to the ultrasoud
image used for the segmentation, the center pixel gets transformed into the
tracking camera coordinate system. Finally the sphere that approximates the
tumor can be drawn into the same coordinate system as the liver surface.
\subsubsection{Automatic 3D}

\subsection{Resection Planning}
For parenchymal-sparing liver resections, the goal is to keep as much healthy tissue as
possible. When the location in the liver and the size of the tumor are known, one can plan a
precise resection from these informations.
% After collecting the needed
% information as described in the previous sections, the planning can 

\section{Workflow}
In this chapter the conceptual workfolw through a liver resection using the
desired system will be presented.
\subsection{Resection planning for non-anatomical ...}

%%% Local Variables:
%%% TeX-master: "MscThesis"
%%% End: