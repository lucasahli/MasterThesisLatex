%%%%%%%%%%%%%%%%%%%%%%%%%%%%%%%%%%%%%%%%%%%%%%%%%%%%%%%%%%%%%%%%%%%%%%%%
% This is the conclusion chapter file.
%%%%%%%%%%%%%%%%%%%%%%%%%%%%%%%%%%%%%%%%%%%%%%%%%%%%%%%%%%%%%%%%%%%%%%%%
%
% Author:   René Widmer
%           Institute for Surgical Technology and Biomechanics ISTB
%           University of Bern
%           rene.widmer@istb.unibe.ch
%
% Date:     10/28/2009
%
%%%%%%%%%%%%%%%%%%%%%%%%%%%%%%%%%%%%%%%%%%%%%%%%%%%%%%%%%%%%%%%%%%%%%%%%

\chapter{Discussion and Conclusions}
% \section{Discussion}
In this project we developed, implemented and tested a new workflow to overcome the registration part in navigated
liver surgeries. By
moving an optically tracked ultrasound probe over the patients liver surface,
the surface gets sampled and afterwards reconstruced using an existing
reconstruction algorithm. The accuracy of the reconstructed surface was investigated in an
experiment with a liver phantom and it showed that the accuracy is compareable to registration based
methods \cite{nam2011automatic}. Tumors can be localized with the
intraoperative ultrasound and are semi-automatically segmented on
a freezed image. They are approximated by a sphere and added to the 3D scene with
the surface 
to create a more complete model of the liver. The accuracy of the tumor
reconstruction is not yet analysed. The created model is used
to plan the resection of the tumor. The planned resection volume is also added to the 3D scene
and can be used to navigate during the surgery. A method was developed
which allows to perform navigated liver operations without registration and the
associated deformation problems. The expensive preoperative model
and the required semi-automatic segmentations are not needed for this new approach.

The workflow compared with other navigation methods in liver surgeries
changed, so a usability test was performed. Its results show that surgeons could
envisage removing tumors with such a system and it is worth to investigate
more time in this kind of projects. 
% \textit{Interpret your results in the context of past and current studies and literature on the same topic. Attempt to explain inconsistencies or contrasting opinion. Highlight the novelty of your work. Objectively discuss the limitations.}

% \section{Conclusions}


In summary, it can be said that an accurate, intraoperative surface reconstruction on anterior
part of the liver was achieved. The boundary of the reconstructed area is
error-prone and should not be trusted. The method is not applicable on the posterior
and inferior part of the liver because it is not reachable with the
intraoperative ultrasound. The presented system is still in a prototypical state
and further development is required.
% which enables for navigated surgeries without registration
% The surface shows 
% Large errors on superior part
% Not applicable on 
% No detection of contact with another organs

% \textit{Formulate clear conclusions which are supported by your research results.}

\endinput
%%% Local Variables:
%%% TeX-master: "MscThesis"
%%% End: